\section{Sequences and Series of Functions}

\setcounter{theorem}{7}
\begin{theorem}[Cauchy Criterion for Uniform Convergence]
  $ \{f_n\} $ converges uniformly with the Cauchy Criterion,
  i.e.
  \begin{flalign*}
    \forall \varepsilon > 0,\,
    \exists N \in \mathbb{N}
    \text{ s.t. }
    \forall n > m > N,\,
    x \in E,
    \left| f_m(x) - f_n(x) \right| < \varepsilon.
  \end{flalign*}
\end{theorem}

\setcounter{theorem}{9}
\begin{theorem}[Weierstrass M Test]
  If $ \left| f_n(x) \right| \leq M_n $
  for all $ n > N $ and  $ x \in E $,
  where $ \sum_{n=1}^{\infty} M_n $ is convergent.
  then $ \{f_n\} $ converges uniformly.
\end{theorem}

\begin{theorem}[Interchange of Limits with Uniform Convergence]\quad

  \begin{enumerate}
    \item Suppose $ X $ is a metric space where
      $ E \subseteq X $ and 
      $ f_n \colon E \to \mathbb{R} $ \textbf{converges uniformly} to $ f \colon E \to \mathbb{R} $.
    \item Also suppose that $ x $ is a limit point of $ E $.
    \item Finally, suppose $ \lim_{t \to x} f_n(t) $ exists $ \forall n $.
  \end{enumerate}

  Then the limits can be interchanged, i.e.
  \begin{flalign*}
    \lim_{n\to \infty} \lim_{t\to x} f_n(x)
    = \lim_{t\to x} \lim_{n\to \infty} f_n(t)
  \end{flalign*}

\end{theorem}

\begin{theorem}[Continuity preserved by uniform convergence]
  If
  \begin{enumerate}
    \item $ f_n \colon E \to \mathbb{R} $ is continuous,
    \item  $ f_n \to f $ uniformly,
  \end{enumerate}

  then $ f $ is continuous on $ E $.
\end{theorem}

\begin{theorem}[Uniform convergence on a compact set]
  If $ K $ is compact and
  \begin{enumerate}
    \item $ f_n \colon K \to \mathbb{R} $ is continuous,
    \item  $ f_n \to f $ pointwise on $ K $,
    \item $ f $ is continuous,
    \item $ f_{n + 1}(x) \leq f_n(x) $ for $ x \in K $,
  \end{enumerate}

  then $ f_n \to f $ uniformly on  $ K $.

  In other words, on a compact domain, 
  a decreasing sequence of continuous functions
  which converge to a continuous function
  converges uniformly.
\end{theorem}

\setcounter{theorem}{14}
\begin{theorem}[$ \mathcal{C}(X) $ is a complete metric space]
\end{theorem}

\newpage
\begin{theorem}[Integrability is preserved by uniform convergence]
  Suppose
  \begin{enumerate}
    \item $ f_n \in \mathcal{R}[a, b] $,
    \item  $ f_n \to f $ uniformly,
  \end{enumerate}

  Then $ f \in \mathcal{R}[a, b] $, and in particular,
  \begin{flalign*}
    \lim_{n\to \infty} \int_{a}^{b} f_n(x) dx
    = \int_{a}^{b} \lim_{n\to \infty} f_n(x) dx.
  \end{flalign*}

  Corollary is that this works for when a series of functions converges uniformly.
\end{theorem}

\begin{theorem}[Differentiability is preserved by uniform convergence]
  Suppose
  \begin{enumerate}
    \item $ f_n $ is differentiable on $ [a, b] $,
    \item  $ f_n \to f $ uniformly for some $ f $ 
    \item $ \frac{d}{dx} f_n(x_0) \to \frac{d}{dx} f(x_0) $ for some $ x_0 \in [a, b] $
      (anchor)
  \end{enumerate}
  \begin{flalign*}
    \lim_{n\to \infty} \frac{d}{dx} f_n(x)
    = \frac{d}{dx} \left( \lim_{n\to \infty} f_n(x) \right)
  \end{flalign*}
\end{theorem}

\begin{theorem}[Existence of continuous but nowhere differentiable function]
  $ \exists $ continuous  $ f\colon \mathbb{R}\to \mathbb{R} $
  s.t.  $ f'(x) $ does not exist  $ \forall x \in \mathbb{R} $.
  \begin{itemize}
    \item Idea is that you keep adding spikes of increasing period,
      have the heights follow a geometric series so it converges.
    \item Then the $ f_n $'s converge uniformly by  $ M $ test.
  \end{itemize}
\end{theorem}

\setcounter{theorem}{22}

\begin{theorem}[Selection Theorem]
  Suppose
  \begin{enumerate}
    \item $ E $ is countable,
    \item  $ f_n \colon E \to \mathbb{C} $ is pointwise bounded
  \end{enumerate}

  Then $ \exists \{f_{n_k}\} $ that is pointwise convergent.

  Bounded implies convergent subsequence applied to entire function.
\end{theorem}

\begin{theorem}[Uniform convergence implies Equicontinuity on a compact domain]
  Suppose
  \begin{enumerate}
    \item $ K $ is compact,
    \item $ f_n\colon K \to \mathbb{R} $ is continuous,
    \item  $ f_n \to f $ uniformly,
  \end{enumerate}

  then $ \{f_n\} $ is equicontinuous.
\end{theorem}

\newpage
\begin{theorem}[Arzel\`{a}-Ascoli Theorem]
  Suppose
  \begin{enumerate}
     \item $ K $ is compact,
     \item $ f_n \colon K \to \mathbb{R} $ where  $ \{f_n\} $ is pointwise bounded and equicontinuous,
  \end{enumerate}
  then 
  \begin{enumerate}
    \item $ \{f_n\} $ is uniformly bounded,
    \item $ \{f_n\} $ has a uniformly convergent subsequence  $ \{f_{n_k}\} $.
  \end{enumerate}

  An equicontinuous and pointwise bounded sequence of functions on a compact domain
  has a uniformly continuous subsequence (and is also uniformly bounded).
\end{theorem}

\begin{theorem}[Weierstrass Theorem]
  Suppose $ f \colon [a, b] \to \mathbb{R} $ is continuous.
  Then $ \exists $ polynomials $ \{P_n\} $ s.t.
  \begin{flalign*}
    \forall \varepsilon > 0,\,
    \exists N \text{ s.t. }
    n \geq N 
    \implies \|P_n - f\| < \varepsilon.
  \end{flalign*}
  (Any continuous function can be uniformly approximated by a sequence of polynomials on a closed interval.)
\end{theorem}

\begin{theorem}[Uniform Closure of an Algebra is an Algebra]
\end{theorem}

\setcounter{theorem}{30}
\begin{theorem}[We can find a function that goes through two points]
  Let $ \mathcal{A} $ be an algebra that separates points and vanishes at no point.
  Then 
  \begin{flalign*}
    \forall x, y \in \mathcal{K} \ (x \neq y),\,
    \forall c, d \in \mathbb{R},\,
    \exists f \in \mathcal{A} \text{ s.t. }
    f(x) = c \text{ and } f(y) = d.
  \end{flalign*}
\end{theorem}

\begin{theorem}[Stone-Weierstrass Theorem]
  The uniform closure of any algebra of continuous functions on a compact set $ K $ which:
  \begin{enumerate}
    \item Separates points,
    \item Vanishes at no points,
  \end{enumerate}

  is $ \mathcal{C}(K) $. 
  The complex case also requires the algebra to be self-adjoint
  ($ f \in \mathcal{A} \implies \overline{f} \in \mathcal{A} $).

  (optimality of $ \mathcal{C}(K) $)
\end{theorem}
