\begin{proof}
  $ \mathcal{F} $ is an algebra if the following three conditions hold:

  \begin{enumerate}
    \item $ \varnothing \in \mathcal{F} $,
    \item $ A \in \mathcal{F} \implies A^c \in \mathcal{F} $,
    \item $ A, B \in \mathcal{F} \implies A \cup B \in \mathcal{F} $.
  \end{enumerate}

  Assume that
  \begin{flalign*}
    A, B \in \mathcal{F} 
    \implies A \setminus B \in \mathcal{F}.
  \end{flalign*}

  First we'll show that (ii) holds.
  Let $ A \in \mathcal{F} $.
  Since $ \Omega \in \mathcal{F} $ as well,
  the given assumption tells us that
  $ \Omega \setminus A = A^c = \mathcal{F} $.

  Note that this implies that (i) holds as well,
  since $ \Omega^c = \varnothing $.

  Finally, we'll show that (iii) holds.
  Note that showing (iii) is sufficient for the proof
  as finite unions follow from pairwise unions via induction (mentioned in class).
  Let $ A, B \in \mathcal{F} $
  and suppose that $ A \cup B \notin \mathcal{F} $.
  Observe that
  \begin{flalign*}
    &A^c \cup A 
    = \Omega \\
    &\implies (A^c \setminus B) \cup (A \cup B) 
    = \Omega \\
    &\implies 
    A \cup B 
    = \Omega \setminus (A^c \setminus B).
  \end{flalign*}

  By the contrapositive of the assumption,
  \begin{flalign*}
    A \cup B
    = \Omega \setminus (A^c \setminus B)
    \notin \mathcal{F}
    \implies 
    \text{either } \Omega \notin \mathcal{F}
    \text{ or } (A^c \setminus B) \notin \mathcal{F},
  \end{flalign*}

  but we know this not to be true,
  since $ \Omega \in \mathcal{F} $ is given,
  and  $ A^c \setminus B \in \mathcal{F} $ 
  by (ii) and the assumption.
  So (iii) holds by contradiction.
  As all conditions hold,
  we conclude that $ \mathcal{F} $ is an algebra.
\end{proof}
