\section{Question 5}

\begin{proof}

  Let $ S = \{E_{\alpha} \colon \alpha \in A\} $.
  Following the hint,
  let $ \mathcal{M} $ be the set of 
  all sets $ E $ that satisfy the property that
  $ \exists $ a countable $ S_E := \{E_{\alpha_j}\colon j = 1, 2, \cdots \} $
  s.t. $ E \in \sigma( S_E ) $,
  where $ \sigma(S_E) $ is the $ \sigma $-algebra generated by $ S_E $. 

  We'll first show that $ \mathcal{M} $ is a $ \sigma $-algebra. 
  Consider $ E = \varnothing $
  and $ S_E = \varnothing $,
  which generates the  $ \sigma $-algebra  $ \sigma(S_E) = \{\varnothing\} $.
  In particular, $ E \in \sigma \left( S_E \right) $
  so $ \varnothing \in \mathcal{M} $ by definition of $ \mathcal{M} $.

  Next let's check that $ \mathcal{M} $ satisfies closure under complements. 
  Let $ E \in \mathcal{M} $.
  By definition of $ \mathcal{M} $,
  $ \exists S_E $ that is countable subcollection of $ S $
  where $ E \in \sigma \left( S_E \right) $.
  But since $ \sigma \left( S_E \right) $ is a  $ \sigma $-algebra,
  $ E^c \in \sigma \left( S_E \right) $.
  In particular,
  this means that for $ E^c $,
  $ S_E $ satisfies the asserted property,
  so $ E^c \in \mathcal{M} $.

  Now we'll verify that $ \mathcal{M} $ is closed under countable unions.
  Let  $ E^1, E^2, \cdots \in \mathcal{M} $
  be a countable list of sets.
  For each $ n = 1, 2, \cdots $,
  $ \exists S_{E^n} $ which is a countable subcollection of $ S $
  s.t. $ E^n \in \sigma \left( S_{E^n} \right) $.
  Consider the $ \sigma $-algebra generated by the countable unions of each  $ S_{E^n} $:
  \begin{flalign*}
    \sigma \left( \bigcup_{i = 1}^{\infty} S_{E^i} \right).
  \end{flalign*}

  Since $ E^n \in \sigma \left( S_{E^n} \right) $,
  $ E^n \in \sigma \left( \bigcup_{i = 1}^{\infty} S_{E^i} \right) $
  for each $ n $
  as $ S_{E^n} \subset \bigcup_{i=1}^{\infty}S_{E^i} $.
  Hence
  \begin{flalign*}
    \bigcup_{n=1}^{\infty} E^n \in \sigma \left( \bigcup_{i = 1}^{\infty} S_{E^i} \right).
    \tag{Closure under countable unions}
  \end{flalign*}

  In particular $ \mathcal{M} $ is a  $ \sigma $-algebra
  that satisfies the asserted property.
  Since $ \mathcal{F} $ is the generated  $ \sigma $-algebra,
  $ \mathcal{F} \subset \mathcal{M} $
  so the asserted property also applies to $ \mathcal{F} $ as required.
  
\end{proof}
