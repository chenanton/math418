\section{Question 2}

\begin{proof}
  We will count the number of pairs $ (A, B) $ 
  where  $ A, B \subset S $ s.t.
  \begin{flalign*}
    A \subset B.
  \end{flalign*}

  First, let's consider $ B $.
  Since we can either include or exclude each element of $ S $
  in a subset,
  there are $ 2^n $ possible subsets of $ S $,
  and in particular there are $ 2^n $ possible
  assignments for $ B $. 
  Furthermore, for $ k \in \mathbb{Z} $ s.t. $ 0 \leq k \leq n $,
  the number of subsets $ B $ that have size $ k $ is $ \binom{n}{k} $.
 
  Next, we count the number of subsets $ A $
  that satisfy  $ A \subset B $ for a given $ B $ with size $ k $.
  Since there are $ 2^{k} $ subsets of $ B $, 
  the number of subset pairs $ (A, B) $ that satisfy $ A \subset B $ is:
  \begin{flalign*}
    \sum_{k=0}^{n} \binom{n}{k}2^k
    = \sum_{k=0}^{n} \binom{n}{k}2^k1^{n-k}
    = (2 + 1)^n 
    = 3^n.
    \tag{Binomial Theorem}
  \end{flalign*}

  We then count the total number of subset pairs $ (A, B) $.
  From above, there are $ 2^n $ possible assignments to $ A $ 
  and $ 2^n $ possible assignments to $ B $.
  Hence the total number of subset pairs $ (A, B) $ is
  \begin{flalign*}
    (2^n)^2 = 4^n.
  \end{flalign*}

  Consider the probability space $ (\Omega, \mathcal{F}, P) $ where $ \Omega = 2^S \times 2^S $,
  $ \mathcal {F} = 2^{\Omega} $, and $ P $ the uniform measure with respect to $ \Omega $ (given).
  From class,
  \begin{flalign*}
    P(A \subset B)
    &= P(\{(A, B) \in \Omega \colon A \subset B\}) \\
    &= \frac{ \left| \{(A, B) \in \Omega \colon A \subset B\} \right| }{ \left| \Omega \right| } \\
    &= \frac{3^n}{4^n} \\
    &= \left( \frac{3}{4} \right)^n.
  \end{flalign*}

\end{proof}
