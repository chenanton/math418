\begin{enumerate}
  \item In order to compute the probability of interest,
    we first count the number of hands containing exactly one pair.
    We'll start by considering values of the cards. 
    There are $ \binom{13}{1} $ ways to choose the value of $ a $,
    and there are $ \binom{12}{3} $ ways to choose the value of $ b $, $ c $, and $ d $. 
    Next we consider the possible suits that the cards can be.
    For cards with the value $ a $, we must choose  $ 2 $ suits,
    so there are  $ \binom{4}{2} $ ways to choose the suits of $ a $.
    Then for each of the cards with values $ b $, $ c $, and $ d $,
    there are $ \binom{4}{1} $ ways to choose the suit of the card.

    Next, count the total number of poker hands by choosing any $ 5 $
    of the $ 52 $ cards. 
    Hence, the probability of interest (say $ p $) is:

    \begin{flalign*}
      p =  \frac{1}{\binom{52}{5}}
      \binom{13}{1}
      \binom{12}{3}
      \binom{4}{2}
      \binom{4}{1}^3
      \approx 0.4226.
    \end{flalign*}

  \item Similar to before, count the number of hands containing exactly two pairs.
    First start by considering values.
    There are $ \binom{13}{2} $ ways to choose the values of  $ a $ and  $ b $,
    and thus there are  $ \binom{11}{1} $ ways to choose the value of $ c $.
    Next, let's look at suits.
    For each of $ a $ and $ b $,
    there are  $ \binom{4}{2} $ ways to choose the suits of the cards,
    and there are  $ \binom{4}{1} $ ways to choose the suit of the card with value of $ c $.
    Similar to above, there are $ \binom{52}{2} $ total poker hands.
    Hence, the probability of interest $ p $ is:
    \begin{flalign*}
      p =  \frac{1}{\binom{52}{5}}
      \binom{13}{2}\binom{11}{1}\binom{4}{2}^2\binom{4}{1}
      \approx 0.04754.
    \end{flalign*}
\end{enumerate}
