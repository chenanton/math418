Let $ E = \{E_{\alpha} \colon \alpha \in A\} $
where $ E \neq \varnothing $.

\begin{proof}

  Let
  \begin{flalign*}
    \sigma \left( E \right) = \bigcap_{\beta} \mathcal{F}_{\beta}
  \end{flalign*}

  where $ \mathcal{F}_{\beta} $ is a $ \sigma $-algebra
  s.t. $ E \subset \mathcal{F}_{\beta} $.
  
  We'll show that $ \sigma(E) $ is indeed a  $ \sigma $-algebra,
  since all $ \mathcal{F}_{\beta} $ are also $ \sigma $-algebras.
  First  $ \varnothing \in \sigma(E) $,
  since  $ \varnothing \in \mathcal{F}_{\beta} $ $ \forall \beta $.
  Similar logic shows that $ \sigma(E) $
  is closed under complements and countable unions, as follows.
  Let $ B \in \sigma(E) $.
  We can be sure that $ B^c \in \sigma(E) $,
  since $ B^c \in \mathcal{F}_{\beta} $  $ \forall \beta $.
  Let $ B_1, B_2, \cdots $ be a countable number of sets.
  We can also be sure that $ \bigcup_{n=1}^{\infty} \in \sigma(E) $
  since $ \bigcup_{n=1}^{\infty} \in \mathcal{F}_{\beta} $
  $ \forall \beta $.
  In particular, $ \sigma(E) $ is a $ \sigma $-algebra.
  

  It is then trivial that $ \sigma(E) $ is the smallest of all 
  $ \sigma $-algebras $ \mathcal{F}_{\beta} $
  s.t. $ E_{\alpha} \in \mathcal{F}_{\beta} $ $ \forall \alpha \in A $
  by definition of intersection.

\end{proof}
